\documentclass{scrartcl}

\usepackage[T1]{fontenc}
\usepackage[utf8]{inputenc}
\usepackage{lmodern}
%\usepackage[scaled=0.85]{beramono}
\usepackage{ucs}
\usepackage{cite}
\usepackage{amsmath}
\usepackage{amsfonts}
\usepackage{amssymb}
\usepackage{graphicx}
\usepackage[ngerman]{babel}
\usepackage{enumerate}
\usepackage{wrapfig}
\usepackage{caption}
\usepackage{listings}
\usepackage{subcaption}
\usepackage{float}
\usepackage{geometry}
\geometry{a0paper, top=0mm, inner=0mm, outer=0mm, bottom=0mm, headsep=0mm, footskip=0mm}
%\usepackage[bold,full]{complexity}
\usepackage{tikz}

\usetikzlibrary{decorations.markings,decorations.pathreplacing}
\usetikzlibrary{shapes.geometric,shapes,automata,positioning}
\usetikzlibrary{shadows,calc}

\renewcommand*\familydefault{\sfdefault} %% Only if the base font of the document is to be sans serif

\newcommand{\parity}{\uplus}

\begin{document}
	
	\def\ND{3.7cm}
	
\begin{tikzpicture}[node distance=\ND,thick, remember picture, overlay]
\tikzstyle{defi}=[rectangle, draw,rectangle split, rectangle split parts=2,text width=8cm,line width=1mm, fill=green!60]
\tikzstyle{defi2}=[rectangle, draw,rectangle split, rectangle split parts=2,text width=8cm,line width=1mm, fill=green!40]
\tikzstyle{satz}=[rectangle, draw,rectangle split, rectangle split parts=2,text width=8cm,line width=1mm, fill=blue!50]
\tikzstyle{satz2}=[rectangle, draw,rectangle split, rectangle split parts=2,text width=8cm,line width=1mm, fill=blue!30]
\tikzstyle{klass}=[]
\tikzstyle{klassNP}=[node distance=5cm]
\tikzstyle{kEdge}=[line width=2mm]
\tikzstyle{sideklass}=[node distance = 11cm,]
% \newcommand{\src}[1]{#1}
\newcommand{\src}[1]{}


% Rahmen 
%\draw [very thick,draw,black] ([xshift=1.5cm,yshift=1.5cm] current page.south west) -- ([xshift=1.5cm,yshift=-3.5cm] current page.north west) -- ([xshift=-1.5cm,yshift=-3.5cm] current page.north east) -- ([xshift=-1.5cm,yshift=1.5cm] current page.south east) -- cycle;

% Titel
\node  [name=titel,anchor=north east] at ([xshift=-1.5cm,yshift=-3.5cm]current page.north east) (komplx) {\fontsize{123}{50}\selectfont Komplexitätstheorie};

\node [font=\normalsize,anchor=north east,align=left] at (komplx.south east) {Poster von Kai Sauerwald. Nach Folien von Prof. Schwentick.};
\tikzstyle{hier}=[node distance=1cm]


%%%%%%%%%%%%%%%%%
%%  Baum
%%%%%%%%%%%%%%%

\fontsize{45}{33}\selectfont
\node (AC0) [klass, anchor=south] at ([xshift=-1cm,yshift=2cm] current page.south) {$\mathbf{AC^0}$} ;
% \node (ACC0) [klass,above of= AC0] {$\mathbf{ACC^0}$} ;
\node (TC0) [klass,above  of= AC0] {$\mathbf{TC^0}$} ;
\node (NC1) [klass,above of= TC0] {$\mathbf{NC^1}$} ;

\node (L) [klass,above of= NC1] {$\mathbf{L}=\mathbf{SL}$} ;
\node (NL) [klass,above  of= L] {$\mathbf{NL}$} ;
\node (LOGCFL) [klass,above  of= NL] {$\mathbf{LOGCFL}$} ;
\node (AC1) [klass,above of= LOGCFL] {$\mathbf{AC^1}$} ;
\node (TC1) [klass,above of= AC1] {$\mathbf{TC^1}$} ;
\node (NC2) [klass,above of= TC1] {$\mathbf{NC^2}$} ;

\node (ACi) [klass,above of= NC2] {$\mathbf{AC^i}$} ;
\node (TCi) [klass,above of= ACi] {$\mathbf{TC^i}$} ;
\node (NCi) [klass,above of= TCi] {$\mathbf{NC^{i+1}}$} ;

\node (NC) [klass,above of= NCi] {$\mathbf{NC}$} ;


\node (P) [klass,above of= NC] {$\mathbf{P=AL}$} ;
\node (ZPP) [klass,above of= P] {$\mathbf{ZPP}$} ;
\node (coRP) [klassNP,above right of= ZPP] {$\mathbf{coRP}$} ;
\node (RP) [klassNP,above left of= ZPP] {$\mathbf{RP}$} ;
\node (BPP) [klassNP,above of= ZPP, node distance=1.8*\ND] {$\mathbf{BPP}$} ;
\node (coNP) [klassNP,above of= coRP] {$\mathbf{coNP}$} ;
\node (NP) [klassNP,above of= RP] {$\mathbf{NP}$} ;

\node (DP) [klass,above of= BPP, node distance=1.5*\ND] {$\mathbf{DP}$} ;
\node (TP2) [klass,above of= DP] {$\mathbf{\Theta^p_2=P^{\Sigma_{1}^p[log]}}$} ;
\node (DP2) [klass,above of= TP2] {$\mathbf{\Delta^p_2=P^{\Sigma_{1}^p}}$} ;
\node (SP2) [klassNP,above left of= DP2] {$\mathbf{\Sigma_2^p}$} ;
\node (PP2) [klassNP,above right of= DP2] {$\mathbf{\Pi_2^p}$} ;
\node (TP3) [klassNP,above left of= PP2] {$\mathbf{\Theta^p_3=P^{\Sigma_{2}^p[log]}}$} ;
\node (DP3) [klass,above of= TP3] {$\mathbf{\Delta^p_3=P^{\Sigma_{2}^p}}$} ;
\node (PH) [klass,above of= DP3] {$\mathbf{PH}$} ;

\node (PSPACE) [klass,above of= PH] {$\mathbf{PSPACE}$} ;
\node (EXPTIME) [klass,above of= PSPACE]{$\mathbf{EXP}$} ;
\node (ELEMENTARY) [klass,above of= EXPTIME]{$\mathbf{ELEMENTARY}$} ;
\node (PR) [klass,above of= ELEMENTARY]{$\mathbf{PR}$} ;
\node (R) [klass,above of= PR]{$\mathbf{R}$} ;
\node (coRE) [klass,above right of= R] {$\mathbf{coRE} $};
\node (RE) [klass,above left of= R] {$\mathbf{RE} $};

\node (AH) [klass,color=gray,above of= R, node distance=2*\ND]{Arithmetische Hierarchie} ;

%\node (REG) [klass, right =of TC0] {$\mathbf{REG}$};
%\node (CFL) [klass,above =of REG] {$\mathbf{CFL}$};

% \node (POLYLOGSPACE) [sideklass,left of= coRP] {$\mathbf{POLYLOGSPACE}$};
%\node (Ppp) [left =of PSPACE] {$\mathbf{P^{PP}=P^{\#P}}$};
%\node (PP) [sideklass,below of= Ppp] {$\mathbf{PP}$};

% \draw (AC0) [kEdge,-|] edge   (ACC0);
% \draw (ACC0)[kEdge] edge  (TC0);

\draw (AC0) [kEdge,-|] edge   (TC0);
\draw (TC0) [kEdge]edge  (NC1);
\draw (NC1) [kEdge]edge  (L);
\draw (L) [kEdge]edge  (NL);
\draw (NL)[kEdge] edge  (LOGCFL);
\draw (LOGCFL)[kEdge] edge  (AC1);
%\draw (CFL) [kEdge]edge  (LOGCFL);
%\draw (REG) [kEdge]edge  (NC1);
%\draw (REG)[kEdge] edge  (CFL);
\draw (AC1) [kEdge]edge  (TC1);
\draw (TC1) [kEdge]edge  (NC2);
\draw (NC2)[kEdge,loosely dashed] edge  (ACi);
\draw (ACi)[kEdge] edge  (TCi);
\draw (TCi) [kEdge]edge  (NCi);
\draw (NCi) [kEdge,loosely dashed]edge  (NC);
\draw (NC) [kEdge]edge  (P);
\draw (P) [kEdge]edge  (ZPP);
\draw (ZPP) [kEdge]edge  (coRP);
\draw (coRP) [kEdge]edge  (coNP);
\draw (coRP) [kEdge]edge  (BPP);
\draw (ZPP) [kEdge]edge  (RP);
\draw (RP) [kEdge]edge  (BPP);
\draw (RP) [kEdge]edge  (NP);
\draw (coNP) [kEdge]edge  (DP);
\draw (NP) [kEdge]edge  (DP);
\draw (DP) [kEdge]edge  (TP2);
\draw (TP2)[kEdge] edge  (DP2);
\draw (DP2) [kEdge]edge  (SP2);
\draw (SP2) [kEdge]edge  (TP3);
\draw (DP2)[kEdge] edge  (PP2);
\draw (PP2)[kEdge] edge  (TP3);
\draw (TP3) [kEdge]edge  (DP3);
\draw (DP3) [kEdge,loosely dashed]edge  (PH);
\draw (PH) [kEdge]edge  (PSPACE);


\draw (PSPACE) [kEdge]edge  (EXPTIME);
\draw (EXPTIME) [kEdge]edge  (ELEMENTARY);
\draw (ELEMENTARY) [kEdge]edge  (PR);
\draw (PR) [kEdge,-|] edge  (R);
\draw (R) [kEdge,-|] edge  (coRE);
\draw (R) [kEdge,-|] edge  (RE);

\draw (AH) [kEdge,color=gray,dashed] edge  (coRE);
\draw (AH) [kEdge,color=gray,dashed] edge  (RE);

%\draw (PH) [kEdge]edge  (Ppp);
%\draw (PP) [kEdge]edge  (Ppp);
\node (PP) [color=black!40] at ([xshift=3cm,yshift=2cm]PP2) {$\mathbf{PP}$};
\draw (PP) [kEdge, bend right=10,color=black!20]edge ([xshift=2cm]PSPACE.south);
\draw (NP) [color=black!20,bend right=30,kEdge]edge  (PP);
\draw (BPP) [color=black!20,bend right=25,kEdge]edge  (PP);
\draw (coNP) [color=black!20,bend right=20,kEdge]edge  (PP);

%\node (time) at (72,17) {$\mathbf{TIME(f)}$};
%\node (coNTime) [hier,above left =of time] {$\mathbf{co-NTIME(f)}$};
%\node (NTime) [hier,above right =of time] {$\mathbf{NTIME(f)}$};
%\node (space) [hier,above right =of coNTime] {$\mathbf{SPACE(f)}$};
%\node (nspace) [hier,above =of space] {$\mathbf{NSPACE(f)=co-NSPACE(f)}$};
%\node (time2e) [hier,above =of nspace] {$\mathbf{TIME(2^{\mathcal{O}(f)})}$};
%\node (space2) [hier,right =of time2e] {$\mathbf{SPACE(f^2)}$};
%
%\draw (time) [kEdge] edge (coNTime) ;
%\draw (time) [kEdge] edge (NTime) ;
%\draw (coNTime) [kEdge] edge (space) ;
%\draw (NTime) [kEdge] edge (space) ;
%\draw (space) [kEdge] edge (nspace) ;
%\draw (nspace) [kEdge] edge (time2e) ;
%\draw (nspace) [kEdge] edge (space2) ;

%%%%%%%%%%%%%%%%
% Probleme
%%%%%%%%%%%%%%%%

\def\probW{30cm}

%AC0 Probleme
%Addition
\node (ADD) [defi,rectangle split parts=1,text width=\probW,anchor=south west] 
at ([xshift=1.5cm, yshift=1.5cm] current page.south west) 
{ 
	\textbf{\textsc{Addition}}  \hfill in $\mathbf{AC^0}$
};
\draw (AC0) [line width=1mm] edge (ADD.east);

%TC0-Vollständige Probleme
%Multiplikation
\node (MULT) [defi2,rectangle split parts=1,text width=\probW,anchor=south west] 
at (ADD.north west) 
{ 
	\textbf{\textsc{Multiplikation}}  \hfill $\mathbf{TC^0}$-C
};
\draw (TC0) [line width=1mm] edge (MULT.east);

%NC1-Vollständige Probleme
%Auswertung aussagenlogischer Formeln
\node (FVP) [defi,text width=\probW,anchor=south west]
at (MULT.north west) 
{ 
	\textbf{\textsc{FormulaValueProblem}}  \hfill $\mathbf{NC^1}$-C\nodepart{second} 
	\textbf{Gegeben:} Variablenfreie boolesche Formel $B$ \newline
	\textbf{Frage:} Ist $B$ wahr? 
};
\draw (NC1) [line width=1mm] edge (FVP.east);

%L-Vollständige Probleme
%UReach
\node (UREACH) [defi2,text width=\probW,anchor=south west] 
at (FVP.north west) 
{ 
	\textbf{\textsc{UREACH} \src{(KT 00/02)}} \hfill $\mathbf{L}$-C  \nodepart{second} 
	\textbf{Gegeben:} Ungerichteter Graph $G=(V,E)$ und zwei Knoten $s,t\in V$ von $G$ \newline
	\textbf{Frage:} Gibt es einen Weg von $s$ nach $t$ in $G$?
};
\draw (L.west) [line width=1mm] edge (UREACH.east);

%NL-Vollständige Probleme
%Reach
\node (REACH) [defi,text width=\probW,anchor=south west] 
at (UREACH.north west) 
{ 
	\textbf{\textsc{REACH} \src{(KT 00/02)}} \hfill $\mathbf{NL}$-C  \nodepart{second} 
	\textbf{Gegeben:} Gerichteter Graph $G=(V,E)$ und zwei Knoten $s,t\in V$ von $G$ \newline
	\textbf{Frage:} Gibt es einen Weg von $s$ nach $t$ in $G$?
};
\draw (NL.west) [line width=1mm] edge (REACH.east);

%LOGCFL-Vollständige Probleme
\node (acyclicCSP) [defi2,text width=\probW,anchor=south west] 
at (REACH.north west) 
{ 
	\textbf{\textsc{AcyclicCS} \src{(KT 00/02)}} \hfill $\mathbf{LOGCFL}$-C  \nodepart{second} 
	\textbf{Gegeben:} Variablenmenge $V$, Domäne $U$, Constraints $C$, so dass $ (V,C) $ azyklisch ist \newline
	\textbf{Frage:} Gibt es eine Lösung $\sigma\subseteq V\times U$, so dass $\sigma(\mathcal{C})$ wahr wird?
};
\draw ([xshift=-2cm] LOGCFL.north) [line width=1mm] edge (acyclicCSP.east);


%NC2 Probleme
%Determinante
\node (DET) [defi,rectangle split parts=1,text width=\probW, anchor=south west] 
at (acyclicCSP.north west) 
{ 
	\textbf{\textsc{Determinante}} \hfill in $\mathbf{NC^2}$
};
\draw (NC2) [line width=1mm] edge [] (DET.east);

% P-Vollständige  Probleme
\node (HORNSAT) [defi2,text width=\probW,anchor=south west]
at (DET.north west) 
{
	\textbf{\textsc{HornSAT}} \hfill $\mathbf{P}$-C  \nodepart{second} 
	\textbf{Gegeben:} Eine Menge von Horn-Klauseln $K$ \newline
	\textbf{Frage:} Ist $K$ erfüllbar?
};
\draw (P) [line width=1mm] edge [] (HORNSAT.east);



% Satz von Rice
%\node (RICE) [defi,text width=\probW, anchor=north west] 
%at  ([xshift=1.5cm,yshift=-3.5cm] current page.north west)
%{
%	\textbf{Satz von Rice} \src{(GTI  18/10)}
%	\nodepart{second} 
%	Für jede nicht triviale Menge $S$ berechenbarer Funktion mit $\emptyset \neq S \neq \mathcal{R}$ gilt:\newline
%	Die Sprache $C(S):=\{w|f_{M_w} \in S\}]$ ist undecc
%	\nodepart{third} 
%	\textbf{Komplexität:} $C(S)$ ist $\mathbf{RE}$-Hart
%};
%\draw (RE) [line width=1mm] edge [] (RICE.west);

\node (TMHALT) [defi,rectangle split parts=1,text width=\probW, anchor=north west] 
at ([xshift=1.5cm,yshift=-3.5cm] current page.north west) 
{ 
	\textbf{\textsc{TM-Halt}} \hfill in $\mathbf{RE}$
};
\draw (RE) [line width=1mm] edge [] (TMHALT.east);

% 2-PlayerCorridorTiling
\node (FOString) [defi2,text width=\probW, anchor=north west] 
at  (TMHALT.south west) 
{
	\textbf{\textsc{SAT-FO-String}}  \hfill in $\mathbf{R}$
	\nodepart{second} 	
%	\begin{description}
%	\item[Gegeben: ] Eine PL Formel $\varphi$ über Strings
%	\item[Frage: ] Ist $\varphi$ erfüllbar?
%	\end{description}
	\textbf{Gegeben:} Eine PL Formel $\varphi$ über Strings \newline
	\textbf{Frage:} Ist $\varphi$ erfüllbar?
};
\draw ([yshift=0cm] FOString.east) [line width=1mm] edge [] (R.west);

% 2-PlayerCorridorTiling
\node (2PTiling) [defi,text width=\probW, anchor=north west] 
at  (FOString.south west) 
{
	\textbf{\textsc{2-PlayerCorridorTiling}}  \hfill $\mathbf{EXP}$-C 
	\nodepart{second} 
	\textbf{Gegeben:} Kachelmenge $T$ und Zahl $k$ (unär) \newline
	\textbf{Frage:} Hat Spieler 1 eine Gewinnstrategie im Tiling-Spiel?
};
\draw ([yshift=-3cm] 2PTiling.east) [line width=1mm] edge [bend right=0] (EXPTIME.west);

%QBF
\node (QBF) [defi2,text width=\probW, anchor=north west] 
at  (2PTiling.south west) 
{
	\textbf{\textsc{QBF}}  \hfill $\mathbf{PSPACE}$-C 
	\nodepart{second} 
	\textbf{Gegeben:} Quantifizierte AL Formel $\varphi$\linebreak (in Pränex-Normalform und KNF)\newline
	\textbf{Frage:} Ist $\varphi$ erfüllbar?
	\nodepart{third}
	\textbf{Komplexität:} PSPACE-Vollständig
};
\draw (QBF.east) [line width=1mm] edge [] (PSPACE);

%2Tiling
\node (2Tiling) [defi,text width=\probW, anchor=north west] 
at  (QBF.south west) 
{
	\textbf{\textsc{CorridorTiling}}  \hfill $\mathbf{PSPACE}$-C 
	\nodepart{second} 
	\textbf{Gegeben:} Kachelmenge $T$ und Zahl $k$ (unär) \newline
	\textbf{Frage:} Gibt es eine $k\times l$ Kachelung?
};
\draw ([yshift=1cm]2Tiling.east) [line width=1mm] edge [] (PSPACE);

%SAT
\node (SAT) [defi2,text width=\probW,anchor=north west]
at (2Tiling.south west) 
{
	\textbf{\textsc{SAT}} \hfill $\mathbf{NP}$-C  \nodepart{second} 
	\textbf{Gegeben:} AL Formel $\varphi$ in KNF \newline
	\textbf{Frage:} Ist $\varphi$ erfüllbar?
};
\draw (SAT.east) [line width=1mm] edge [] (NP);

% NP-Tiling
\node (NPtiling) [defi,text width=\probW,anchor=north west]
at (SAT.south west) 
{
	\textbf{\textsc{Tiling}} \hfill $\mathbf{NP}$-C  \nodepart{second} 
	\textbf{Gegeben:} Kachelmenge $T$ und Zahlen $c,r$ (unär) \newline
	\textbf{Frage:} Gibt es eine $c\times r$-Lösung für $T$?
};
\draw (NPtiling.east) [line width=1mm] edge [] (NP);

% TSP
\node (TSP) [defi2,text width=\probW,anchor=north west]
at (NPtiling.south west) 
{
	\textbf{\textsc{TSP}} \hfill $\mathbf{NP}$-C  \nodepart{second} 
	\textbf{Gegeben:} Entfernungsgraph $G,\ k\in \mathbb{N}$ \newline
	\textbf{Frage:} Hat $G$ einen Kreis durch alle Knoten mit Gewicht $\leq k$?
};
\draw (TSP.east) [line width=1mm] edge [] (NP);


% \node [defi] at (\leftPr+10,45) 
% {
%   \textsc{TSP} (GTI 21/08)  \nodepart{second} 
%   \textbf{Gegeben:} Entfernungsgraph $G,\ k\in \mathbb{N}$
%   \newline
%   \textbf{Frage:} Hat $G$ einen Kreis durch alle Knoten mit Gewicht $\leq k$?
%   \nodepart{third}
%   \textbf{Komplexität:} NP-Vollständig
% };


%%%%%%%%%%%%%%%%
% Sätze
%%%%%%%%%%%%%%%%
\def\satzW{32cm}
\def\satzWmin{33cm}


\node (TimeHierarchie) [satz,text width=\satzW, minimum width=\satzWmin, anchor=south east] 
at ([xshift=-1.5cm,yshift=1.5cm] current page.south east) 
{
	\textbf{Zeithierarchie-Satz} \hfill Hartmanis, Stearns  \nodepart{second} 
	Seien $f,g$ zeitkonstruierbar mit 
	$f(n)\in\Omega(n)$ und $g(n)\in\omega(f(n)\log (f(n)))$, so gilt: \[\mathbf{TIME(f)\subsetneq TIME(g)}\]
};

\node (SpaceHierarchie) [satz2,text width=\satzW, minimum width=\satzWmin, anchor=south east] 
at (TimeHierarchie.north east) 
{
	\textbf{Platzhierarchie-Satz} \hfill Stearns, Szepietowski  \nodepart{second} 
	Seien $f,g$ zeitkonstruierbar mit 
	$f(n)\in\Omega(\log n)$ und $g(n)\in\omega(f(n))$, so gilt:
	\[\mathbf{SPACE(f)\subsetneq SPACE(g)}\]
	\[\mathbf{NSPACE(f)\subsetneq NSPACE(g)}\]
};

\node (SatzSavitch) [satz,text width=\satzW, minimum width=\satzWmin, anchor=south east] 
at (SpaceHierarchie.north east) 
{
	\textbf{Satz von Savitch} \hfill Savitch  \nodepart{second} 
	Sei $S$ platzkonstruierbar und $S(n)\geq \log n$, so gilt:
	$$ \mathbf{NSPACE(S)\subseteq SPACE(S^2)}$$
};

\node (SatzImmerman) [satz2,text width=\satzW, minimum width=\satzWmin, anchor=south east] 
at (SatzSavitch.north east) 
{
	\textbf{Satz} \hfill Immerman, Szelepcsényi  \nodepart{second}
	Für jedes platzkonstruierbare $S$ mit $S(n)\geq\log n$ gilt:
	$$ \mathbf{NSPACE(S)}=\mathbf{coNSPACE(S)} $$
};

\node (SatzAlternation) [satz,text width=\satzW, minimum width=\satzWmin, anchor=south east] 
at (SatzImmerman.north east) 
{
	\textbf{Satz} \hfill Chandra, Kozen, Stockmeyer   \nodepart{second} 
	\begin{enumerate}[(a)]
	\item Ist $S(n)=\Omega(\log n)$, so gilt $\mathbf{ASPACE(S)\subseteq TIME(2^{\mathcal{O}(S)})}$
	\item Ist $T(n)=\Omega(n)$, so gilt $\mathbf{TIME(T)\subseteq ASPACE(\log T)}$
	\end{enumerate}
};

\filldraw [fill=red!30,line width=1mm] ([xshift=0.5\pgflinewidth,yshift=-0.5\pgflinewidth] SatzAlternation.north west) -- ([xshift=0.5\pgflinewidth,yshift=62cm-0.5\pgflinewidth] SatzAlternation.north west) -- ([xshift=-0.5\pgflinewidth,yshift=62cm-0.5\pgflinewidth] SatzAlternation.north east) -- ([xshift=-0.5\pgflinewidth,yshift=-0.5\pgflinewidth]SatzAlternation.north east) -- cycle ;

%%%%%%%%%%%%%%%%
% NPO Hierarchie
%%%%%%%%%%%%%%%%
\fontsize{45}{33}\selectfont

\node (optimierung) [rotate=90, anchor=north west] at (SatzAlternation.north west) {\fontsize{55}{33}\selectfont \textbf{Optimierung}};

\fontsize{35}{25}\selectfont

\node (PO) [anchor=south east] at ([xshift=4cm] optimierung.south west) {$\mathbf{PO}$} ;
\node (FPTAS) [hier,node distance=1.5cm,above =of PO] {$\mathbf{FPTAS}$} ;
\node (PTAS) [hier,node distance=1.5cm,above =of FPTAS] {$\mathbf{PTAS}$} ;
\node (APX) [hier,node distance=1.5cm,above =of PTAS] {$\mathbf{APX}$} ;
\node (NPO) [hier,node distance=1.5cm,above =of APX] {$\mathbf{NPO}$} ;

\draw (PO) edge [kEdge,-|] (FPTAS);
\draw (FPTAS) edge [kEdge,-|] (PTAS);
\draw (PTAS) edge [kEdge,-|] (APX);
\draw (APX) edge [kEdge,-|] (NPO);


\coordinate (APX2NPO) at ($ (NPO) !.5! (APX) $);
\coordinate (PTAS2APX) at ($ (APX) !.5! (PTAS) $);
\coordinate (FPTAS2PTAS) at ($ (PTAS) !.5! (FPTAS) $);
\coordinate (PO2FPTAS) at ($ (FPTAS) !.5! (PO) $);
\node (optTSP)        [anchor=east] at ([xshift=11.5cm] APX2NPO) {\textbf{\textsc{TSP}}};
\node (optMaxSAT)     [anchor=east] at ([xshift=11.5cm] PTAS2APX) {\textbf{\textsc{MAX-SAT}}};
\node (optScheduling) [anchor=east] at ([xshift=11.5cm] FPTAS2PTAS) {\textbf{\textsc{Small-Scheduling}}};
\node (optRucksack)   [anchor=east] at ([xshift=11.5cm] PO2FPTAS) {\textbf{\textsc{Rucksack}}};
 
\draw (APX2NPO) edge [kEdge,loosely dashed] (optTSP.west);
\draw (PTAS2APX) edge [kEdge,loosely dashed] (optMaxSAT.west);
\draw (FPTAS2PTAS) edge [kEdge,loosely dashed] (optScheduling.west);
\draw (PO2FPTAS) edge [kEdge,loosely dashed] (optRucksack.west);



%%%%%%%%%%%%%%%%%%%%%%%%%%%
% Satz von TODA
%%%%%%%%%%%%%%%%%%%%%%%%%%%%
\fontsize{37}{33}\selectfont

\node (ClassOP) [satz, fill=blue!40,text width=15cm, anchor=south east]
at ([yshift=-1mm]SatzAlternation.north east) 
{
	\textbf{\fontsize{35}{33}\selectfont Satz: Klassenoperatoren} \nodepart{second} 
	\fontsize{35}{33}\selectfont Sei $\mathcal{C}$ unter Turing-Reduktion abgeschloseen. Dann gilt:
	\fontsize{37}{33}\selectfont
	{\boldmath\begin{align*}
		\mathbf{BP}\cdot\mathbf{BP}\cdot\mathcal{C} & \subseteq\mathbf{BP}\cdot\mathcal{C}\\
		\oplus\cdot\oplus\cdot\mathcal{C}           & \subseteq\oplus\mathcal{C}\\
		\oplus\cdot\mathbf{BP}\cdot\mathcal{C}      & \subseteq\mathbf{BP}\cdot\oplus\cdot\mathcal{C}\\
%		\oplus\cdot\mathbf{BP}\cdot\mathcal{C}      & \subseteq\mathbf{BP}\cdot\oplus\cdot\mathcal{C}\\
		\exists\cdot\mathcal{C}                     & \subseteq\mathbf{BP}\cdot\oplus\cdot\mathcal{C}\\
		\exists\cdot\mathbf{BP}\cdot\mathcal{C}     & \subseteq\mathbf{BP}\cdot\exists\cdot\mathcal{C}\\
		\exists\cdot\mathbf{BP}\cdot\mathbf{P}      & =\mathbf{MA}\\
		\mathbf{BP}\cdot\exists\cdot\mathbf{P}      & =\mathbf{AM}
		\end{align*}}
};

\node (fPvsNP) [anchor=north west] at (ClassOP.north west -| optimierung.north east) {
	\fontsize{30}{33}\selectfont\textbf{Falls\boldmath\ $\mathbf{P}\neq \mathbf{NP}$}};

\node (fPvsNP) [anchor=north east, text width=7cm, align=right] at ([xshift=0cm,yshift=0cm] ClassOP.north west) {\fontsize{25}{10}\selectfont Trennendes\\ Beispiel};

%%%%%%%%%%%%%%%%
% Beweissysteme
%%%%%%%%%%%%%%%%


\fontsize{35}{25}\selectfont

%\filldraw [fill=red!30] (SatzAlternation.north west) -- ([yshift=16cm] SatzAlternation.north west) -- ([yshift=16cm] SatzAlternation.north east) -- (SatzAlternation.north east) -- cycle ;

\node (proofsystem) [rotate=90, anchor=north west] at (ClassOP.north west-| SatzAlternation.north west) {
	\fontsize{55}{33}\selectfont \textbf{Beweissysteme}};

\node (npchar) [anchor=south west] at ([xshift=2.5cm,yshift=0cm]proofsystem.north west)  {$\mathbf{NP}=\mathbf{PCP(log,poly)}=\mathbf{PCP(0,poly)}=\mathbf{PCP(log,1)}$};
\node (mBPP) at ([xshift=1cm,yshift=1cm] npchar.north west) {$\mathbf{BPP}$};
\node (MA) at ([xshift=4cm,yshift=3cm] npchar.north west) {$\mathbf{MA}$};
\node (AM) at ([yshift=2.5cm] MA.north) {$\mathbf{AM}$};
\node (Spi2) at ([yshift=2.5cm] AM.north) {$\mathbf{\Pi^p_2}$};
\node (IP) at ([xshift=-2cm,yshift=2.5cm] Spi2.north) {$\mathbf{PSPACE}=\mathbf{IP}$};

\draw (mBPP) edge [kEdge] (MA);
\draw ([xshift=-8cm]npchar.north) edge [kEdge] (MA);

\draw (MA) edge [kEdge] (AM);
\draw (AM) edge [kEdge] (Spi2);
\draw (Spi2) edge [kEdge] (IP);


% Das PCP Theorem
\node (PCP) [satz, fill=yellow!40,text width=22cm, anchor=north east]
at ([yshift=16cm-1mm]ClassOP.north east) 
{
	\textbf{PCP} \textit{(probabilistically checkable proof)} \nodepart{second} 
	Die Klasse $PCP(\mathcal{F},\mathcal{G})$ der Sprachen $L$ für die es einen Verifizierer $V$ gibt der
	\begin{itemize}
	\item $\leq r(n)$ Zufallsbits verwendet, \hfill($r\in\mathcal{F}$)
	\item $\leq q(n)$ Bits des Beweises ließt, \hfill($q\in\mathcal{G}$)
	\item für einen Beweis $\pi$ alles aus $L$ akzeptiert,
	\item und für jeden Beweis kein $x\notin L$ mit W'keit $>\frac{1}{2}$ akzeptiert.
	\end{itemize}
};

\coordinate (temp2) at (ClassOP.north west -| SatzAlternation.north west);

\draw [line width=1mm] ([xshift=0.5\pgflinewidth,yshift=-0.5\pgflinewidth] ClassOP.north west)  -- (temp2);


%%%%%%%%%%%%%%
% W-Hierarchie
%%%%%%%%%%%%%%

\fontsize{35}{25}\selectfont

\node (Parametrisierung) [rotate=90, anchor=north west, rectangle split, rectangle split parts=2, rectangle split part align={left, left}] at ([yshift=21cm] proofsystem.north west) {\fontsize{55}{33}\selectfont \textbf{Beschreibungskomplex.} \nodepart{second}\fontsize{55}{33}\selectfont\textbf{\& Parametrisierung}};

\node (fpt) [anchor=south west] at ([xshift=1.3cm,yshift=0.5cm]Parametrisierung.south west) {$\mathbf{FPT}$};
\node (w1) [hier,above =of fpt] {$\mathbf{W[1]=A[1]}$};
\node (w2) [hier,above =of w1] {$\mathbf{W[2]}$};
\node (wSAT) [node distance=2cm,above =of w2] {$\mathbf{W[SAT]}$};
\node (wP) [hier,above =of wSAT] {$\mathbf{W[P]}$};
\node (paraNP) [hier,above =of wP] {$\mathbf{paraNP}$};

\node (aStar) [node distance=1.5cm,right  =of wP] {$\mathbf{AW[*]}$};
\node (a2) [node distance=1.7cm,below =of aStar] {$\mathbf{A[2]}$};
\node (XP) [hier,above =of aStar] {$\mathbf{XP}$};

\draw (fpt) edge [kEdge] (w1);
\draw (w1) edge [kEdge] (w2);
\draw (w2) edge [kEdge] (a2);
\draw (w2) edge [kEdge,loosely dashed] (wSAT);
\draw (wSAT) edge [kEdge] (wP);
\draw (wP) edge [kEdge] (paraNP);

\draw (w1) edge [kEdge] (a2);
\draw (a2) edge [kEdge,loosely dashed] (aStar);
\draw (aStar) edge [kEdge] (XP);

%\draw (wSAT) edge [kEdge] (aStar);
\draw (wP) edge [kEdge] (XP);

% Das Parametrisierte Probleme
\node (paramProbl) [satz, fill=yellow!40,text width=\satzW, minimum width=\satzWmin, anchor=north east]
at ([yshift=62cm] SatzAlternation.north east) 
{
	\textbf{Parametrisiertes Problem} \nodepart{second} 
	Ein Tupel $(Q,\kappa)$ heißt parametrisiertes Problem, falls $Q\subseteq\Sigma^\ast$ und $\kappa:\Sigma^\ast\to\mathbb{N}$ in polynomieller Zeit berechenbar ist.
};

%%%%%%%%%%%%%%
% Logik & Komplexität
%%%%%%%%%%%%%%


\def\mydist{0.8cm}

\node (lAC0D) [anchor=south east] at ([xshift=-.0cm, yshift=-20cm]paramProbl.south east) {$\mathbf{FO(+,\times)}$};
\node (lREGD) [anchor=south east] at ([yshift=-.4cm]lAC0D.north east) {$\mathbf{MSO}$};
\node (lTC0D) [anchor=south east] at ([xshift=0cm, yshift=\mydist]lAC0D.north east) {$\mathbf{FO(+,\times)+C}$};
\node (lLD) [anchor=south east] at ([xshift=0cm, yshift=\mydist]lTC0D.north east) {$\mathbf{DTC}$};
\node (lNLD) [anchor=south east] at ([xshift=0cm, yshift=\mydist]lLD.north east) {$\mathbf{TC}$};
\node (lPD) [anchor=south east] at ([xshift=0cm, yshift=\mydist]lNLD.north east) {$\mathbf{HornESO, LFP}$};
\node (lNPD) [anchor=south east] at ([xshift=0cm, yshift=\mydist]lPD.north east) {$\mathbf{ESO}$};
\node (lPSPACED) [anchor=south east] at ([xshift=0cm, yshift=\mydist]lNPD.north east) {$\mathbf{PFP}$};

\node (lAC0) [anchor=center] at ([xshift=-12cm]lAC0D.east) {$\mathbf{AC^0}$};
\node (lREG) [anchor=center] at ([xshift=-15cm]lREGD.east) {$\mathbf{REG}$};
\node (lTC0) [anchor=center] at ([xshift=-12cm]lTC0D.east) {$\mathbf{TC^0}$};
\node (lL) [anchor=center] at ([xshift=-12cm]lLD.east) {$\mathbf{L}$};
\node (lNL) [anchor=center] at ([xshift=-12cm]lNLD.east) {$\mathbf{NL}$};
\node (lP) [anchor=center] at ([xshift=-12cm]lPD.east) {$\mathbf{P}$};
\node (lNP) [anchor=center] at ([xshift=-12cm]lNPD.east) {$\mathbf{NP}$};
\node (lPSPACE) [anchor=center] at ([xshift=-12cm]lPSPACED.east) {$\mathbf{PSPACE}$};


\node (ltitel) [anchor=north east] at ([xshift=-.0cm, yshift=-0cm]paramProbl.south east) {\fontsize{50}{33}\selectfont \textbf{Komplexität \& Logik}};

\draw (lAC0) edge [kEdge] (lTC0);
\draw (lTC0) edge [kEdge] (lL);
\draw (lREG) edge [kEdge] (lL);
\draw (lL) edge [kEdge] (lNL);
\draw (lNL) edge [kEdge] (lP);
\draw (lP) edge [kEdge] (lNP);
\draw (lNP) edge [kEdge] (lPSPACE);

\draw (lAC0) edge [kEdge,loosely dashed] (lAC0D);
\draw (lTC0) edge [kEdge,loosely dashed] (lTC0D);
\draw (lREG) edge [kEdge,loosely dashed] (lREGD);
\draw (lL) edge [kEdge,loosely dashed] (lLD);
\draw (lNL) edge [kEdge,loosely dashed] (lNLD);
\draw (lP) edge [kEdge,loosely dashed] (lPD);
\draw (lNP) edge [kEdge,loosely dashed] (lNPD);
\draw (lPSPACE) edge [kEdge,loosely dashed] (lPSPACED);

% Logiken und Komplexitätsklassen
\node (logikACompl) [satz, fill=yellow!40,text width=\satzW, minimum width=\satzWmin, anchor=south east]
at ([yshift=-1mm] PCP.north east) 
{
	\textbf{Komplexität einer Logik} \nodepart{second} 
	Eine Logik $\mathcal{L}$ hat die Komplexität $\mathcal{C}$, falls für jede geschlossene Formel $\varphi\in\mathcal{L}$ gilt, dass $ \{\text{enc}(\mathfrak{A})\ |\ \mathfrak{A}\mathfrak\models\varphi) \}\in\mathcal{C} $.
	
};

\coordinate (temp) at (logikACompl.north west -| ltitel.north west);

\draw [line width=1mm] ([xshift=.3cm]ltitel.north west) -- ([xshift=.3cm]temp);

%%%%%%%%%%%%%%%%%%%%%%%
% Kleine Hierarchie
%%%%%%%%%%%%%%%%%%%%%%%
\fontsize{30}{25}\selectfont

\filldraw [fill=blue!50,line width=1mm] ([xshift=0.5\pgflinewidth,yshift=-0.5\pgflinewidth]TSP.south west) -- ([xshift=-0.5\pgflinewidth,yshift=-0.5\pgflinewidth]TSP.south east) -- ([xshift=-0.5\pgflinewidth,yshift=0.5\pgflinewidth]HORNSAT.north east) -- ([xshift=0.5\pgflinewidth,yshift=0.5\pgflinewidth]HORNSAT.north west) -- cycle ;
\node (time) [anchor=south] at (HORNSAT.north) {$\mathbf{TIME(f)}$};
\node (coNTime) [hier,above left =of time] {$\mathbf{coNTIME(f)}$};
\node (NTime) [hier,above right =of time] {$\mathbf{NTIME(f)}$};
\node (space) [hier,above right =of coNTime] {$\mathbf{SPACE(f)}$};
\node (nspace) [hier,above =of space] {$\mathbf{NSPACE(f)}=\mathbf{coNSPACE(f)}$};
\node (time2e) [hier,above =of nspace] {$\mathbf{TIME(2^{\mathcal{O}(f)})}$};
\node (space2) [hier,right =of time2e] {$\mathbf{SPACE(f^2)}$};

\draw (time) [kEdge] edge (coNTime) ;
\draw (time) [kEdge] edge (NTime) ;
\draw (coNTime) [kEdge] edge (space) ;
\draw (NTime) [kEdge] edge (space) ;
\draw (space) [kEdge] edge (nspace) ;
\draw (nspace) [kEdge] edge (time2e) ;
\draw (nspace) [kEdge] edge (space2) ;


\fontsize{35}{25}\selectfont

\node (sn) [anchor=north west] at (TSP.south west) {\textbf{Falls\boldmath\ $f(n)\geq\log(n)$}};

\fontsize{55}{33}\selectfont
\node (littleOverview) [rotate=90, anchor=south west,align=left] at (HORNSAT.north east) {\fontsize{55}{33}\selectfont \textbf{Kleine}\\\textbf{Hierarchie}};

\end{tikzpicture}
	
\end{document}
