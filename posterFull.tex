\documentclass{scrartcl}

\usepackage[T1]{fontenc}
\usepackage[utf8]{inputenc}
\usepackage{lmodern}
%\usepackage[scaled=0.85]{beramono}
\usepackage{ucs}
\usepackage{cite}
\usepackage{amsmath}
\usepackage{amsfonts}
\usepackage{amssymb}
\usepackage{graphicx}
\usepackage[ngerman]{babel}
\usepackage{enumerate}
\usepackage{wrapfig}
\usepackage{caption}
\usepackage{listings}
\usepackage{subcaption}
\usepackage{float}
\usepackage{geometry}
\geometry{a0paper, top=0mm, inner=0mm, outer=0mm, bottom=0mm, headsep=0mm, footskip=0mm}
%\usepackage[bold,full]{complexity}
\usepackage{tikz}

\usetikzlibrary{decorations.markings,decorations.pathreplacing}
\usetikzlibrary{shapes.geometric,shapes,automata,positioning}
\usetikzlibrary{shadows}

\newcommand{\parity}{\uplus}

\begin{document}


\begin{tikzpicture}[node distance=4.1cm,thick]

\tikzstyle{defi}=[rectangle, draw,rectangle split, rectangle split parts=2,text width=8cm,line width=1mm, fill=green!20]
\tikzstyle{satz}=[rectangle, draw,rectangle split, rectangle split parts=2,text width=8cm,line width=1mm, fill=blue!20]
\tikzstyle{klass}=[]
\tikzstyle{klassNP}=[node distance=5cm]
\tikzstyle{kEdge}=[line width=2mm]
\tikzstyle{sideklass}=[node distance = 11cm,]
% \newcommand{\src}[1]{#1}
\newcommand{\src}[1]{}

\def\leftPr{37.5}
\def\rightPr{18.5}

\draw [very thick,draw,black] (0,0) -- (0,112) -- (84,112) -- (84,0) -- (0,0);




\node [font=\Huge] at (0,0) {};
\node [font=\Huge] at (83,110) {};
\node [font=\Huge] at (83,0) {};
\node [font=\Huge] at (0,110) {};


\fontsize{100}{33}\selectfont
\node (titel) [anchor=north east] at (84,112) {Komplexitätstheorie};

\fontsize{45}{33}\selectfont
\node (AC0) [klass] at (25,1) {$\mathbf{AC^0}=FO$} ;
% \node (ACC0) [klass,above of= AC0] {$\mathbf{ACC^0}$} ;
\node (TC0) [klass,above  of= AC0] {$\mathbf{TC^0}$} ;
\node (NC1) [klass,above right of= TC0] {$\mathbf{NC^1}$} ;

\node (L) [klass,above  right  of= NC1] {$\mathbf{L}=\mathbf{SL}$} ;
\node (NL) [klass,above  of= L] {$\mathbf{NL}$} ;
\node (LOGCFL) [klass,above right of= NL] {$\mathbf{LOGCFL}$} ;
\node (AC1) [klass,above right  of= LOGCFL] {$\mathbf{AC^1}$} ;
\node (TC1) [klass,above  right  of= AC1] {$\mathbf{TC^1}$} ;
\node (NC2) [klass,above of= TC1] {$\mathbf{NC^2}$} ;

\node (ACi) [klass,above of= NC2] {$\mathbf{AC^i}$} ;
\node (TCi) [klass,above left of= ACi] {$\mathbf{TC^i}$} ;
\node (NCi) [klass,above right of= TCi] {$\mathbf{NC^{i+1}}$} ;

\node (NC) [klass,above of= NCi] {$\mathbf{NC}$} ;


\node (P) [klass,above of= NC] {$\mathbf{P=AL}$} ;
\node (ZPP) [klass,above of= P] {$\mathbf{ZPP}$} ;
\node (coRP) [klassNP,above left of= ZPP] {$\mathbf{coRP}$} ;
\node (RP) [klassNP,above right of= ZPP] {$\mathbf{RP}$} ;
\node (coNP) [klassNP,above of= coRP] {$\mathbf{coNP}$} ;
\node (BPP) [klassNP,above right of= coRP] {$\mathbf{BPP}$} ;
\node (NP) [klassNP,above of= RP] {$\mathbf{NP}$} ;

\node (DP) [klass,above left of= coNP] {$\mathbf{DP}$} ;
\node (TP2) [klass,above of= DP] {$\mathbf{\Theta^p_2=P^{\Sigma_{1}^p[log]}}$} ;
\node (DP2) [klass,above of= TP2] {$\mathbf{\Delta^p_2=P^{\Sigma_{1}^p}}$} ;
\node (SP2) [klassNP,above left of= DP2] {$\mathbf{\Sigma_2^p}$} ;
\node (PP2) [klassNP,above right of= DP2] {$\mathbf{\Pi_2^p}$} ;
\node (TP3) [klassNP,above left of= PP2] {$\mathbf{\Theta^p_3=P^{\Sigma_{2}^p[log]}}$} ;
\node (DP3) [klass,above of= TP3] {$\mathbf{\Delta^p_3=P^{\Sigma_{2}^p}}$} ;
\node (PH) [klass,above of= DP3] {$\mathbf{PH}$} ;

\node (PSPACE) [klass,above of= PH] {$\mathbf{PSPACE}$} ;
\node (EXPTIME) [klass,above of= PSPACE]{$\mathbf{EXPTIME}$} ;
\node (ELEMENTARY) [klass,above of= EXPTIME]{$\mathbf{ELEMENTARY}$} ;
\node (PR) [klass,above of= ELEMENTARY]{$\mathbf{PR}$} ;
\node (R) [klass,above of= PR]{$\mathbf{R}$} ;
\node (coRE) [klass,above right of= R] {$\mathbf{coRE} $};
\node (RE) [klass,above left of= R] {$\mathbf{RE} $};


\node (REG) [klass, below right of= NC1] {$\mathbf{REG}$};
\node (CFL) [klass,right of= L,node distance=5cm] {$\mathbf{CFL}$};

% \node (POLYLOGSPACE) [sideklass,left of= coRP] {$\mathbf{POLYLOGSPACE}$};
\node (Ppp) [right =of PSPACE] {$\mathbf{P^{PP}=P^{\#P}}$};
\node (PP) [sideklass,below of= Ppp] {$\mathbf{PP}$};

% \draw (AC0) [kEdge,-|] edge   (ACC0);
% \draw (ACC0)[kEdge] edge  (TC0);

\draw (AC0) [kEdge,-|] edge   (TC0);
\draw (TC0) [kEdge]edge  (NC1);
\draw (NC1) [kEdge]edge  (L);
\draw (L) [kEdge]edge  (NL);
\draw (NL)[kEdge] edge  (LOGCFL);
\draw (LOGCFL)[kEdge] edge  (AC1);
\draw (CFL) [kEdge]edge  (LOGCFL);
\draw (REG) [kEdge]edge  (NC1);
\draw (REG)[kEdge] edge  (CFL);
\draw (AC1) [kEdge]edge  (TC1);
\draw (TC1) [kEdge]edge  (NC2);
\draw (NC2)[kEdge,dashed] edge  (ACi);
\draw (ACi)[kEdge] edge  (TCi);
\draw (TCi) [kEdge]edge  (NCi);
\draw (NCi) [kEdge,dashed]edge  (NC);
\draw (NC) [kEdge]edge  (P);
\draw (P) [kEdge]edge  (ZPP);
\draw (ZPP) [kEdge]edge  (coRP);
\draw (coRP) [kEdge]edge  (coNP);
\draw (coRP) [kEdge]edge  (BPP);
\draw (ZPP) [kEdge]edge  (RP);
\draw (RP) [kEdge]edge  (BPP);
\draw (RP) [kEdge]edge  (NP);
\draw (coNP) [kEdge]edge  (DP);
\draw (NP) [kEdge]edge  (DP);
\draw (coNP)[kEdge] edge  (PP);
\draw (BPP) [kEdge]edge  (PP);
\draw (NP) [kEdge]edge  (PP);
\draw (DP) [kEdge]edge  (TP2);
\draw (TP2)[kEdge] edge  (DP2);
\draw (DP2) [kEdge]edge  (SP2);
\draw (SP2) [kEdge]edge  (TP3);
\draw (DP2)[kEdge] edge  (PP2);
\draw (PP2)[kEdge] edge  (TP3);
\draw (TP3) [kEdge]edge  (DP3);
\draw (DP3) [kEdge,dashed]edge  (PH);
\draw (PH) [kEdge]edge  (PSPACE);


\draw (PSPACE) [kEdge]edge  (EXPTIME);
\draw (EXPTIME) [kEdge]edge  (ELEMENTARY);
\draw (ELEMENTARY) [kEdge]edge  (PR);
\draw (PR) [kEdge] edge  (R);
\draw (R) [kEdge] edge  (coRE);
\draw (R) [kEdge] edge  (RE);

% \draw (P) [kEdge]edge  (POLYLOGSPACE);
% \draw (POLYLOGSPACE) [kEdge, bend left=30]edge  (PSPACE);

\draw (PH) [kEdge]edge  (Ppp);
\draw (PP) [kEdge]edge  (Ppp);
\draw (PP) [kEdge]edge  (PSPACE);

\fontsize{35}{33}\selectfont
% % Andere Probleme:
% 
\node (RICE) [defi,text width=24cm, anchor=north west] at (0,112) 
{
	\textbf{Satz von Rice} \src{(GTI  18/10)}
	\nodepart{second} 
	Fuer jede nicht triviale Menge $S$ berechenbarer Funktion mit $\emptyset \neq S \neq \mathcal{R}$ gilt:\newline
	Die Sprache $C(S):=\{w|f_{M_w} \in S\}]$ ist undecc
	\nodepart{third} 
	\textbf{Komplexit\"at:} $C(S)$ ist $\mathbf{RE}$-Hart
};
\draw (RE) [line width=1mm] edge [] (RICE.east);

% RE (RekursiveEnummerable ->Semi-Entscheidbar) Probleme
% 
% % Menge aller Strings
% \node [defi,text width=23cm] at (\rightPr,90) 
% {
%   \textbf{"\textsc{K}"} \src{(GTI  18/10)}
%   \nodepart{second} 
%   $K$ ist die Menger aller $\{0,1\}$-Strings $\gamma$, für die die Turingmascheine $M_\gamma$ die Eingabe $\gamma$ akzeptiert
%   \nodepart{third} 
%   \textbf{Komplexit\"at:} RE (GTI  18/10)
% };

% % Halteproblem
% \node [defi] at (\rightPr,95) 
% {
%   \textsf{(Allgemeines) Halteproblem} $H$ (GTI  18/19)  
%   \nodepart{second}  
%   $H:=\{w\#x|M_w(x)\text{ h\"alt an}\}$\newline
%   $H_0:=\{w|M_w(\varepsilon)\text{ h\"alt an}\}$
%    \nodepart{third}
%   \textbf{Komplexit\"at:} RE (Offensichtlich)
% };
% 
% %PSPACE-Vollst\"andige Probleme
% 
%CooridorTiling
\node [defi,text width=24cm, anchor=north west] at (0,80) 
{
	\textsc{CooridorTriling} \hfill $\mathbf{PSPACE}$-C
	\nodepart{second} 
	\textbf{Gegeben:} Kachelmenge $T$ und Zahl $c$ \newline
	\textbf{Frage:} Gibt es eine $c\times r$-L\"osung f\"ur $T$?
};

%NFA-Containment
\node [defi,text width=24cm, anchor=north west] at (0,70) 
{
	\textsc{NFA-Conteinment} \hfill $\mathbf{PSPACE}$-C
	\nodepart{second} 
	\textbf{Gegeben:} NFAs $\mathcal{A}_1,\mathcal{A}_2$ \newline
	\textbf{Frage:} Ist $L(\mathcal{A}_1)\subseteq L(\mathcal{A}_2)$?
};


%QBF
\node [defi,text width=24cm, anchor=north west] at (0,60) 
{
	\textsc{QBF}  \hfill $\mathbf{PSPACE}$-C 
	\nodepart{second} 
	Die Menge der wahren quantifizierten aussagen-
	logischen Formeln (in Pr\"anex-Normalform), de-
	ren quantorenfreie Matrix in KNF ist.
	\nodepart{third}
	\textbf{Komplexit\"at:} PSPACE-Vollst\"andig
};

%NP-Vollst\"andige Probleme

% %CliqueO
% \node [defi] at (\leftPr+10,60) 
% {
%   \textsc{CliqueO} (GTI 21/04)  \nodepart{second} 
%   \textbf{Gegeben:} Ungerichteter Graph $G=(V,E)$
%   \newline
%   \textbf{Gesucht:} Maximale Clique in $G$, d.h.: maximale Menge C von Knoten, die paarweise durch Kanten verbunden sind
%   \nodepart{third}
%   \textbf{Komplexit\"at:} NP-Vollst\"andig
% };
% 
% %HamiltonKreis
% \node [defi] at (\leftPr+10,55) 
% {
%   \textsc{HamiltonKreis} (GTI 21/04)  \nodepart{second} 
%   \textbf{Gegeben:} Ungerichteter Graph $G$
%   \newline
%   \textbf{Frage:} Gibt es eine geschlossene Kantenfolge in $G$, die jeden Knoten genau einmal besucht? 
%   \nodepart{third}
%   \textbf{Komplexit\"at:} NP-Vollst\"andig
% };
% 
%SAT
\node [defi,text width=24cm, anchor=north west] at (0,52) 
{
	\textsc{(3-)SAT} \hfill NP-C  \nodepart{second} 
	\textbf{Gegeben:} Aussagenlogische Formel $\varphi$ in KNF (mit je 3 Literalen je Klausel)
	\newline
	\textbf{Frage:} Ist $\varphi$ erf\"ullbar?
};

%TSP
% \node [defi] at (\leftPr+10,45) 
% {
%   \textsc{TSP} (GTI 21/08)  \nodepart{second} 
%   \textbf{Gegeben:} Entfernungsgraph $G,k\in \mathbb{N}$
%   \newline
%   \textbf{Frage:} Hat $G$ einen Kreis durch alle Knoten mit Gewicht $\leq k$?
%   \nodepart{third}
%   \textbf{Komplexit\"at:} NP-Vollst\"andig
% };
% 
% %Col
% \node [defi] at (\leftPr,60) 
% {
%   \textsc{Col} (GTI 21/01)  \nodepart{second} 
%   \textbf{Gegeben:} Ungerichteter Graph $G$, Zahl $k$
%   \newline
%   \textbf{Frage:} Lassen sich die Knoten von $G$ mit $k$ Fragen zul\"assig f\"arben.
%   \nodepart{third}
%   \textbf{Komplexit\"at:} NP-Vollst\"andig
% };
% 
% %RucksackO
% \node [defi,text width=23cm] at (\leftPr,55) 
% {
%   \textsc{RucksackO} \src{(GTI 21/02)}  \nodepart{second} 
%   \textbf{Gegeben:} Gewichtsschranke $G$ und $m$ Gegenst\"ande repr\"asentiert durch Werte $w_1,\ldots,w_m$ und Gewichte $g_1,\ldots,g_m$) \newline
%   \textbf{Gesucht:} $I\subseteq \{ 1,\ldots, m\}$, so dass $\sum_{i\in I} w_i$ maximal ist und $\sum_{i\in I} g_i \leq G$ gilt.
%   \nodepart{third}
%   \textbf{Komplexit\"at:} NP-Vollst\"andig
% };
% 
%Tiling
\node [defi,text width=23cm,anchor=north west] at (0,90) 
{
	\textsc{Tiling} \hfill $\mathbf{NP}$-C \nodepart{second} 
	\textbf{Gegeben:} Kachelmenge $T$ und Zahlen $c,r$ (un\"ar kodiert!) \newline
	\textbf{Frage:} Gibt es eine $c\times r$-L\"osung f\"ur $T$?
	\nodepart{third}
	\textbf{Komplexit\"at:} NP-Vollst\"andig (KT 02/08)
};

% MinCycleO
% \node [defi,text width=23cm] at (\leftPr,45) 
% {
%   \textbf{\textsc{MinCycleO}}  \nodepart{second} 
%    \textbf{Gegeben:} Ungerichteter zusammenh\"angender Graph $G=(V,E)$ und Funktion $l:E\to\mathbb{align}$ \newline
%   \textbf{Gesucht:} Kreis $K\subseteq E$ durch alle Knoten mit minimalem Gesamtgewicht $\sum_{e\in K} l(e)$ (oder $\bot$)
%   \nodepart{third}
%   \textbf{Komplexit\"at:} NP-Vollst\"andig (GTI 20/07)
% };
% 
% P  Probleme
% \node (MinSpanningTreeO) [defi,text width=23cm] at (70,33) 
% {
%   \textbf{\textsc{MinSpanningTreeO}} \hfill $\mathbf{P}$  \nodepart{second} 
%   \textbf{Gegeben:} Ungerichteter zusammenh\"angender Graph $G=(V,E)$ und Funktion $l:E\to\mathbb{N}$ \newline
%   \textbf{Gesucht:} Aufspannender Baum $T\subseteq E$ von $G$ mit minimalem Gesammtgewicht $\sum_{e\in T} l(e)$ 
% };
% \draw (P) [line width=1mm] edge [] (MinSpanningTreeO.west);



% P-Vollst\"andige  Probleme
\node (HORNSAT) [defi,text width=24cm,anchor=north west] at (0,43) 
{
	\textbf{\textsc{HornSAT}} \hfill $\mathbf{P}$-C  \nodepart{second} 
	\textbf{Gegeben:} Eine Menge von Horn-Klauseln $K$ \newline
	\textbf{Frage:} Ist $K$ erfuellbar?
};
\draw (P) [line width=1mm] edge [] (HORNSAT);

%NC2 Probleme

%Addition
\node (DET) [defi,rectangle split parts=1,text width=15cm] at (50,27) 
{ 
	\textbf{\textsc{Determinante}} \hfill $\mathbf{NC^2}$
};
\draw (NC2) [line width=1mm] edge [] (DET.west);

%LOGCFL-Vollst\"andige Probleme

%Addition
\node (CNFCFGWORD) [defi,rectangle split parts=1,text width=15cm] at (50,17) 
{ 
	\textbf{\textsc{CNFCFGWORD}}  \hfill $\mathbf{LOGCFL}$
};
\draw (LOGCFL) [line width=1mm] edge (CNFCFGWORD.west);

%NL-Vollst\"andige Probleme

%Reach
\node (REACH) [defi,text width=25cm,anchor=south west] at (0,20) 
{ 
	\textbf{\textsc{REACH} \src{(KT 00/02)}} \hfill $\mathbf{NL}$-C  \nodepart{second} 
	\textbf{Gegeben:} Gerichteter Graph $G=(V,E)$ und zwei Knoten $s,t\in V$ von $G$ \newline
	\textbf{Frage:} Gibt es einen Weg von $s$ nach $t$ in $G$?
};
\draw (NL) [line width=1mm] edge (REACH.east);

%L-Vollst\"andige Probleme

%UReach
\node (UREACH) [defi,text width=25cm,anchor=south west] at (0,13) 
{ 
	\textbf{\textsc{UREACH} \src{(KT 00/02)}} \hfill $\mathbf{L}$-C  \nodepart{second} 
	\textbf{Gegeben:} Ungerichteter Graph $G=(V,E)$ und zwei Knoten $s,t\in V$ von $G$ \newline
	\textbf{Frage:} Gibt es einen Weg von $s$ nach $t$ in $G$?
};
\draw (L) [line width=1mm] edge (UREACH.east);

%NC1-Vollst\"andige Probleme

%Auswertung aussagenlogischer Formeln
\node (FVP) [defi,text width=21.2cm,anchor=west] at (0,10) 
{ 
	\textbf{\textsc{FormulaValueProblem}}  \hfill $\mathbf{NC^1}$-C\nodepart{second} 
	\textbf{Gegeben:} Var. freie boolesche Formel $B$ \newline
	\textbf{Frage:} Ist $B$ wahr? 
};
\draw (NC1) [line width=1mm] edge (FVP.east);

%TC0-Vollst\"andige Probleme

%Multiplikation
\node (MULT) [defi,rectangle split parts=1,text width=17cm,anchor=south west] at (0,4) 
{ 
	\textbf{\textsc{Multipikation}}  \hfill $\mathbf{TC^0}$-C
};
\draw (TC0) [line width=1mm] edge (MULT.east);

%AC0 Probleme

%Addition
\node (ADD) [defi,rectangle split parts=1,text width=17cm,anchor=south west] at (0,0) 
{ 
	\textbf{\textsc{Addition}}  \hfill $\mathbf{AC^0}$
};
\draw (AC0) [line width=1mm] edge (ADD.east);


\node (TimeHierarchie) [satz,text width=24cm, anchor=south east] at (59,0) 
{
	\textbf{Zeithierachie-Satz} \hfill Hartmanis, Stearns  \nodepart{second} 
	Seien $f,g$ zeitkonstruierbar mit \newline
	$f(n)\in\Omega(n)$ und $g(n)\in\omega(f(n)\log (f(n)))$ .\newline
	So gilt: $\mathbf{TIME(f)\subsetneq TIME(g)}$
};

\node (SpaceHierarchie) [satz,text width=24cm, anchor=south east] at (84,0) 
{
	\textbf{Platzhierarchie-Satz} \hfill Stearns, Szepietowski  \nodepart{second} 
	Seien $f,g$ zeitkonstruierbar mit \newline
	$f(n)\in\Omega(\log n)$ und $g(n)\in\omega(f(n))$. So gilt
	\begin{itemize}
	\item $\mathbf{SPACE(f)\subsetneq SPACE(g)}$
	\item $\mathbf{NSPACE(f)\subsetneq NSPACE(g)}$
	\end{itemize}
};

\node (SatzSavitch) [satz,text width=15cm, anchor=south east] at (59,6.5) 
{
	\textbf{Satz von Savitch} \hfill Savitch  \nodepart{second} 
	Sei $S$ platzkonstruierbar \newline und $S(n)\geq \log n$, so gilt
	$$ \mathbf{NSPACE(S)\subseteq SPACE(S^2)}$$
};

\node (SatzImmerman) [satz,text width=24cm, anchor=south east] at (84,8.5) 
{
	\textbf{Satz} \hfill Immerman, Szelepcsenyi  \nodepart{second}
	F\"ur jedes platzkonstruierbare $S$ mit $S(n)\geq\log n$ gilt:
	$$ \mathbf{NSPACE(S)=co-NSPACE(S)} $$
};

\node (SatzAlternation) [satz,text width=24cm, anchor=south east] at (84,33) 
{
	\textbf{Satz} \hfill Chandra, Kozen, Stockmeyer   \nodepart{second} 
	\begin{enumerate}[(a)]
	\item Ist $S(n)=\Omega(\log n)$, so gilt $\mathbf{ASPACE(S)\subseteq TIME(2^{\mathcal{O}(S)})}$
	\item Ist $S(n)=\Omega(n)$, so gilt $\mathbf{TIME(S)\subseteq ASPACE(2^{\log S})}$
	\end{enumerate}
};

\node (SatzOp) [satz,text width=19cm, anchor=south east] at (84,95) 
{
	\textbf{Satz}   \nodepart{second} 
	Sei $\mathcal{C}$ unter Turing-Reduktion abgeschlossen. Dann gilt:
	\begin{enumerate}[(a)]
	\item $\mathbf{BP\cdot BP\cdot \mathcal{C}\subseteq BF\cdot \mathcal{C}}$
	\item $\mathbf{\parity\cdot\parity\cdot\mathcal{C}\subseteq \parity\cdot \mathcal{C}}$
	\item $\mathbf{\parity\cdot BP\cdot\mathcal{C}\subseteq BP\cdot \parity\cdot \mathcal{C}}$
	\item $\mathbf{BP\cdot\mathcal{C}}$ und $\parity\cdot\mathbf{\mathcal{C}}$ sind unter Turing-Reduktion abgeschlossen
	\end{enumerate}
};

\node [font=\normalsize,anchor=south east,text width=8cm,align=right] at (84,0) {Poster von Kai Sauerwald\\ Nach Folien von Prof. Schwentick};
\tikzstyle{hier}=[node distance=1cm]

\node (time) at (72,17) {$\mathbf{TIME(f)}$};
\node (coNTime) [hier,above left =of time] {$\mathbf{co-NTIME(f)}$};
\node (NTime) [hier,above right =of time] {$\mathbf{NTIME(f)}$};
\node (space) [hier,above right =of coNTime] {$\mathbf{SPACE(f)}$};
\node (nspace) [hier,above =of space] {$\mathbf{NSPACE(f)=co-NSPACE(f)}$};
\node (time2e) [hier,above =of nspace] {$\mathbf{TIME(2^{\mathcal{O}(f)})}$};
\node (space2) [hier,right =of time2e] {$\mathbf{SPACE(f^2)}$};

\draw (time) [kEdge] edge (coNTime) ;
\draw (time) [kEdge] edge (NTime) ;
\draw (coNTime) [kEdge] edge (space) ;
\draw (NTime) [kEdge] edge (space) ;
\draw (space) [kEdge] edge (nspace) ;
\draw (nspace) [kEdge] edge (time2e) ;
\draw (nspace) [kEdge] edge (space2) ;


\node (fPvsNP) [rotate=90] at (66,77) {Falls $\mathbf{P\neq NP}$ gilt:};
\node (PO) at (70,70) {$\mathbf{PO}$} ;
\node (FPTAS) [hier,above =of PO] {$\mathbf{FPTAS}$} ;
\node (PTAS) [hier,above =of FPTAS] {$\mathbf{PTAS}$} ;
\node (APX) [hier,above =of PTAS] {$\mathbf{APX}$} ;
\node (NPO) [hier,above =of APX] {$\mathbf{NPO}$} ;
\draw (PO) edge [kEdge,-|] (FPTAS);
\draw (FPTAS) edge [kEdge,-|] (PTAS);
\draw (PTAS) edge [kEdge,-|] (APX);
\draw (APX) edge [kEdge,-|] (NPO);


\node (fpt) at (70,50) {$\mathbf{FPT}$};
\node (w1) [hier,above =of fpt] {$\mathbf{W[1]=A[1]}$};
\node (w2) [hier,above =of w1] {$\mathbf{W[2]}$};
\node (wSAT) [node distance=2cm,above =of w2] {$\mathbf{W[SAT]}$};
\node (wP) [hier,above =of wSAT] {$\mathbf{W[P]}$};
\node (paraNP) [hier,above =of wP] {$\mathbf{paraNP}$};

\node (a2) [node distance=2cm,right =of w2] {$\mathbf{A[2]}$};
\node (aStar) [node distance=2cm,right  =of wP] {$\mathbf{AW[*]}$};
\node (XP) [hier,above =of aStar] {$\mathbf{XP}$};

\draw (fpt) edge [kEdge] (w1);
\draw (w1) edge [kEdge] (w2);
\draw (w2) edge [kEdge,->] (a2);
\draw (w2) edge [kEdge,dashed] (wSAT);
\draw (wSAT) edge [kEdge] (wP);
\draw (wP) edge [kEdge] (paraNP);

\draw (w1) edge [kEdge] (a2);
\draw (a2) edge [kEdge,dashed] (aStar);
\draw (aStar) edge [kEdge] (XP);

\draw (wSAT) edge [kEdge,->] (aStar);
\draw (wP) edge [kEdge,->] (XP);
\end{tikzpicture}
\end{document}